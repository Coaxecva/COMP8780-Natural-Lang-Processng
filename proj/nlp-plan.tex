\documentclass[12pt]{article}
\usepackage{graphicx}
\usepackage{amsfonts}
\usepackage{amsthm}
\usepackage{amsmath}
\usepackage{blkarray}
\usepackage{multirow}
\usepackage[labelfont=bf]{caption}
\usepackage[margin=1.5in]{geometry}
\usepackage{color}
\usepackage{algorithmic}
\usepackage{algorithm}

\begin{document}

\title{NLP7-8780 Project: \textbf{Latent Semantic Analysis for Classifying Genomic Sequences}}
\author{Instructor: Dr. Vasile Rus \\ Student: Quang Tran}
\date{Spring 2017}
\maketitle


\section{Introduction}

\subsection{Latent Semantic Analysis}
Latent Semantic Analysis (LSA) is a technique in natural language processing, in particular distributional semantics, of analyzing relationships between a set of documents and the terms they contain by producing a set of concepts related to the documents and terms. In the field of information retrieval, LSA techniques are used to identify text
documents with related underlying concepts. In this project, we extend these approaches to infer the biological structure through which a collection of organisms are related; we would like to identify and partition evolutionarily similar genomic sequences.

\subsection{Genomic K-mer Sequences}
The term k-mer typically refers to all possible subsequences of length k that are contained in a larger sequence. For instance, we can say ATGCCCATTA is a k-mer of length 10 or a "10-mer". And this "10-mer" contains the 4-mers: ATGC, TGCC, GCCC, CCCA, CCAT, CATT, and ATTA. Since the nucletotides in DNA contain four different bases: A, T, G, and C, these 4-mers above are just a small subset of the $4^k = 4^4 = 256$ possible k-mers.

\subsection{Bag-of-k-mers Model}
A DNA sequences can be hypothesized by some statistical presentation of the k-mers from which it was comprised. LSA techniques are based on the bag-of-words model for prepresenting text documents By treating k-mers as the words, we modify the original model to bag-of-k-mers model in genetic language. 

\section{Project Plan}

\begin{itemize}
	\item Week 1: Study LSA, implement LSA in python
	\item Week 2: Collect genomic data, compute k-mers on data
	\item Week 3: Integrate LSA methods to biological data
	\item Week 4,5,6: Implement and improve the integration
	\item Week 7: Finalize the proposed method, revise project report
	\item Week 8: Review overall
\end{itemize}

\nocite{vrus2017}
\nocite{martin2000speech}
\nocite{landauer1998introduction}
\nocite{dumais2004latent}
 
\bibliography{test}
\bibliographystyle{ieeetr}

%\begin{thebibliography}{9}

%\bibitem{Baeza-Yates99} 
%Ricardo Baeza-Yates and Berthier Ribeiro-Neto. 
%\textit{Modern Information Retrieval}. 
%1999.

%\end{thebibliography}


\end{document}
